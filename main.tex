\documentclass{article}

\usepackage{subcaption}
\usepackage{xparse}
\usepackage{etoolbox}
\usepackage[italian]{babel}
\usepackage[utf8x]{inputenc}
\usepackage{graphicx}
\usepackage{titlesec}
\usepackage{commons}
\usepackage{float}
\usepackage[svgnames]{xcolor}
\usepackage[tikz]{bclogo}
\usepackage{url}



\NewDocumentCommand{\qb}{}{%
    q.b.
}

\makeatletter

\NewDocumentCommand{\warning}{m}{%
\begin{bclogo}[logo=\bcattention, couleurBarre=orange, noborder=true, couleur=white]{Attenzione!}
    {#1}
\end{bclogo}
}

\NewDocumentCommand{\note}{m}{%
\begin{bclogo}[logo=\bcinfo, couleurBarre=cyan, noborder=true, couleur=white]{Nota}
    {#1}
\end{bclogo}
}

\NewDocumentCommand{\setRecipePersonNumber}{m}{%
    \DeclareDocumentCommand{\setRecipePersonNumber@value}{}{#1}%
}

\NewDocumentCommand{\generalRecipeInfos}{}{%
    La ricetta è per \setRecipePersonNumber@value{} persone ed è realizzabile in X ore.%
}

\newcommand{\ingredienti@rec}{\@ifnextchar\bgroup{\ingredienti@consume}{\ingredienti@end}}

\newcommand{\ingredienti@consume}[2]{%
    {#1} & {#2} \\%
    \ingredienti@rec%
}

\newcommand{\ingredienti@end}{%
        \end{tabular}%
    \end{table}%
}

\newcommand{\ingredienti}[2]{%
    \begin{table}[h]%
        \centering
        \begin{tabular}{cl}%
            Ingrediente & Quantità \\ \hline
            \ingredienti@consume{#1}{#2}
}

\makeatother

\author{Massimo Bono}
\title{Il Ricettario\\Ricette gustose per palati Bresciani}
\date{\today}

\NewDocumentCommand{\sectionbreak}{}{%
    \clearpage%
}

%add bibliograph yto table of content
%\addcontentsline{toc}{section}{Bibliography}

\begin{document}
    \maketitle
    \clearpage
    \newpage

    \tableofcontents
    \clearpage
    \newpage

    \section{Introduzione}

Questo ricettario è l'insieme di tutte le ricette che ho scritto eseguite in un certo tempo nella mia vita. Il ricettario è tutto fuorché completo.
    \section{Brasato con polenta}

\setRecipePersonNumber{4}
\generalRecipeInfos{}

\ingredienti%
    {chiodi di garofano}{4 o 6 chiodi}{alloro}{4 o 6 foglie}{sale}{quanto basta}%
    {funghi}{600 grammi di funghi tra chiodini e porcini}%
    {vino rosso}{per sfumare}%
    {carota}{2}%
    {sedano}{2 gambo}%
    {cipolla}{1}%
    {polpa di pomodoro}{4 bicchieri (o 20 cucchiai)}%
    {olio oliva extra vergine}{quanto basta}

\begin{enumerate}

    \item Tritare le carote, le cipolle ed il sedano e metterle da parte;
    \item Mettere  mezzo bicchiere d'olio d'oliva in una capiente pentola di terracotta;
    \item prendere il brasato ed infarinarlo; quindi metterlo in pentola;
    \item inserire mezzo trito di carote, cipolle e sedano;
    \item accendere la fiamma a fuoco alto e lasciar rosolare la carna;
    \item appena rosolato, sfumare il tutto con del vino rosso;
    \item sumato il vino, inserire 2 bicchieri di polpa di pomodoro, 2 chiodi di garofano, 2 foglie di alloro tagliate a metà (in modo che rilascino più sapore), e tanta acqua fino a \textbf{quasi} coprire l'intero brasato (il brasato dovrebbe rimanere fuodi dall'acqua appena appena). Inserire quindi 2 pizzichi di sale ognuno lungo un intero giro di pentola. Infine coprire la pentola con il coperchio (sempre in terracotta); 
    \item 
        Aspettare che si cuocia. Questo processo può durare fino a 2 ore e mezza;
        \insertImage{src/imgs/brasato/06}{0.4}
    \item quando la carne è a metà cottura, togliere il brasato, quindi tagliarlo a fettine di spessore 0.8 - 1 cm. Lo si fa per farla cuocere meglio e per rilasciare più sapore. Per capire quando la carne è a metà cottura bisogna che la carne sia bianca, sugosa e fibrosa all'interno. Dio solito serve un'ora per essere a metà cottura;
    \item 
        nel frattempo prendere un'altra padella, e far rosolare l'altra metà di trito con dell'olio;
        \insertImage{src/imgs/brasato/02}{0.4}
    \item 
        inserire i funghi sgocciolati (non devono essere secchi ma nemmeno carichi di acqua) nella padella, insieme a sale, 2 bicchieri di pomodoro, 2 foglie di alloro tagliate a metà, 2 chiodi di garofano ed una sfilata di olio. Coprire con il coperchio e far cuocere per 15- 20 minuti;
        \insertImage{src/imgs/brasato/07}{0.4}
    \item Il brasato deve avere del sughetto che è ottenuto dall'acqua. Quando l'acqua sta per arrivare al livello corretto di sughetto (che lo decidi te quanto "sugoso" lo vuoi) inserire i funghi;
    \item aspettare 10 minuti;
\end{enumerate}

    \section{Polenta}

Ricetta del papà.

\setRecipePersonNumber{4}
\generalRecipeInfos{}

\ingredienti%
    {acqua fredda}{1.5L}%
    {farina polenta}{3.5hg}%
    {cucchiaio di olio}{1}%
    {cucchiaini da caffé di sale}{2.5}%


\begin{itemize}
    \item metti l'acqua nella polentiera;
    \item inserisci tutta la farina e mescola a fiamma accesa;
    \item muovi la farina dentro la polentiera con un cucchiaio fino a quando non si è addensato;
    \item metti il sale;
    \item aggiungi l'olio;
    \item fai partire la macchinetta per 30/40 minuti;
\end{itemize}
    \section{Pasta al tonno rossa}

\setRecipePersonNumber{2}
\generalRecipeInfos{}

\ingredienti%
    {pasta a tua scelta}{400g}%
    {scatole tonno rio mare grandi}{2}%
    {polpa di pomodoro mutti}{1 scatola grande}%
    {prezzemolo}{\qb{}}%
    {sale}{\qb{}}%
    {pepe nero}{\qb{}}%
    {cannella}{\qb{}}%
    {olio}{\qb{}}%
    {cipolla}{\qb{}}%
    {acqua}{\qb{}}

\begin{enumerate}
    \item prendere una padella, mettere l'olio e qualche rondella di cipolla, quindi far rosolare finché la cipolla non è dorata;
    \item aggiungere il tonno senza romperlo troppo, la pola di pomodoro e dell'acqua (circa 1 bicchiere e mezzo);
    \item aggiungere sale, pepe e cannella (la cannella circa 3 cucchiaini);
    \item posizionare il fuoco a livello "basso" quindi aspettare che l'acqua sfumi quasi del tutto (ci vuole circa 40 minuti);
    \item alla fine inserire una spruzzata di prezzemolo;
    \item nel frattempo accendere l'acqua della pasta, aspettare che bolla, quindi versare la pasta desiderata;
    \item quando la pasta è pronta scolarla ed unirla con il sugo (che non deve essere asciutto, un pò di acqua deve essere rimasta per fare il "sughetto").
\end{enumerate}
    
    \section{Pomodori Ripieni}

\setRecipePersonNumber{2}
\generalRecipeInfos{}

\ingredienti%
    {scatola di tonno grande (o 2 piccole)}{1}%
    {capperi}{10/12}%
    {tubetto di maionese}{0.5}%
    {succo di limone (opzionale)}{\qb{}}%
    {sale}{\qb{}}%
    {Pomodori Lunghi}{4}%
    {Foglie di basilico}{8}%

\warning{Serve la preparazione il giorno prima!}

\subsection{Giorno Prima}
\begin{enumerate}
    \item Prendo i pomodori, li tagli in 2 per lungo;
    \item Pulisco i pomodori tirando fuori tutto e lasciando solo la \quotes{buccia};
    \item mettere i pomodori in un piatto e lasciarli nel frigorifero;
\end{enumerate}

\note{
    Se si ha poco tempo, è anche possibile metterli in frigorifero la mattina per usarli a pranzo. Questo passaggio serve per fare in modo che l'acqua esca naturalmente dai pomodori in modo che poi essi non siano duri da mangiare, bensì morbidi.
}

\subsection{Giorno corrente}
\begin{enumerate}
    \item Mettere tonno e capperi dentro un tritatore;
    \item tritare il mix fino a che non è diventato una crema;
    \item tiro fuori il mix dal tritatore e aggiungo, se voglio, il succo di limone;
    \item Tiro fuori i pomodori preparati precedentemente;
    \item asciugo i pomodori dall'acqua in eccesso;
    \item Riempio i pomodori con l'impasto;
    \item Aggiungo per ogni pomodoro una foglia di basilico;
    \item Servire freddo.
\end{enumerate}
    \section{Crostini con Paté d'Olive}

\setRecipePersonNumber{6}
\generalRecipeInfos{}

\ingredienti%
    {scatola di tonno piccole}{3}%
    {olive verdi snocciolate}{circa 30}%
    {capperi}{1 cucchiaio da minestra}%
    {acciughe}{5-6}%
    {pane (meglio baguette)}{\qb{}}

\begin{enumerate}
    \item Inserire il tonno, olive, capperi, acciughe in un tritatore e tritare;
    \insertImage{src/imgs/crostini-con-pate-olive/01}{0.4}
    \item Tritare fino a che il mix non diventa una crema;
    \insertImage{src/imgs/crostini-con-pate-olive/02}{0.4}
    \item Affettare del pane;
    \insertImage{src/imgs/crostini-con-pate-olive/04}{0.4}
    \item Tostare il pane qualche minuto in forno;
    \item Spalmare il mix sul pane a piacere;
    \item Service caldo o tiepido;
    \insertImage{src/imgs/crostini-con-pate-olive/04}{0.4}
\end{enumerate}


    \section{Riso Pollo e Mandorle}

\setRecipePersonNumber{2}
\generalRecipeInfos{}

\ingredienti%
    {Riso Basmati}{}%
    {Petto di Pollo}{400g}%
    {sale}{\qb{}}%
    {pepe}{\qb{}}%
    {olio extravergine d'oliva}{\qb{}}%
    {miele}{\qb{}}%
    {salsa di soia}{\qb{}}%
    {succo di limone}{\qb{}}%
    {zucchina grande}{1}%
    {carote}{2}%
    {funghi}{\qb{}}%
    {peperoni}{\qb{}}%
    {Mandorle}{\qb{}}%

\note{la ricetta coinvolge 3 fasi in parallelo: cottura del riso, cottura del pollo e cuttura delle verdure.}

\subsection{Cottura dell'acqua}

\begin{enumerate}
    \item far bollire dell'acqua e metterci il sale;
    \item Butta il riso nell'acqua;
\end{enumerate}

\subsection{Pollo}

\begin{enumerate}
    \item Prendere una padella ed inserire il pollo a pezzi di 1-2 cm insieme a sale, olio e pepe;
    \item Una volta che il pollo è cotto aggiungere il 2-3 cucchiaini di miele, 2-3 cucchiaini di soia ed 1 cucchiaini di limone;
\end{enumerate}

\subsection{Verdure}

\begin{enumerate}
    \item Prendere una padella, oliarla ed inserire tutte le verdure considerata (zucchine, peperoni, mandorle); Non inserire il sale poiché esso rovinerebbe le zucchine;
    \item Una volta che le verdure sono cotte aggiungere in padelle sale, pepe, miele, salsa di soia e limone;
    \item mischia il tutto nella padella;
    \item aggiungi mezzo bicchiere d'acqua per far finire la cottura delle verdure;
    \item Aspetta che l'acqua si asciughi;
\end{enumerate}

\subsection{Fase finale}

\begin{enumerate}
    \item Una volta che l'acqua delle verdure si è asciugata unisci le verdure con il pollo e falli andare insieme per alcuni minuti;
    \item Unisci il composto con il riso;
    \item Servire il tutto;
\end{enumerate}



    \section{Limoncello}

\setRecipePersonNumber{2}
\generalRecipeInfos{}

\ingredienti%
    {Alcol 90\%}{250g}%
    {Scorza edibile di limoni}{100g}%
    {zucchero}{125g}%
    {acqua}{250g}%

\warning{La ricetta dura nelle fasi iniziali da un giorno a 14 giorni.}

In realtà le dosi hanno delle proporzioni fisse\cite{bressanini-2018a}. Se $g_{scorza}$ sono i grammi di scorza edibile ricavati dai limoni (in media un limone genera 40g di scorza edibile), allora:

$$g_{alcol} = 2.5 \times g_{scorza}$$
$$g_{zucchero} = 0.5 \times g_{alcol}$$
$$g_{acqua} = g_{alcol}$$

\begin{enumerate}
	\item sciacquare i limoni (per evitare che impurità entrino nel limoncello);
	\item tagliare le scorze usando un pela patate facendo attenzione a non incorporare l'albedo (parte bianca); Per togliere ancora più albedo puoi usare un coltello ed abrasare via l'albedo sulla scorza;
	\item tagliare le striscie di scorza in quadratine per migliorare le reazioni chimiche successive;
	\item inserire la scorza in una bottiglia di vetro, quindi aggiungere l'alcol; conservare quindi il composto per un 1 giorno in un posto lontano dalla luce; In realtà
		questa attesa può durare da 1 ad 3 giorni: per un limoncello pù intenso e profumato scegliere 1 giorno, per un limoncello più mordibo 3 giorni (i limoncelli commerciali solitamente usano 3 giorni);
		In generale meglio 1 giorno;
	\item dopo l'attesa riprendere la bottiglia di limoncello, aggiungere zucchero e mischiare;
	\item aggiungere quindi subito dopo l'acqua;

		\insertImage{src/imgs/limoncello/01}{0.4}
	\item volendo è possibile berlo subito; tuttavia per avere un limoncello con più sapori aspettare da 1 a 3 settimane;
\end{enumerate}


    \section{Zucchine grigliate}

\setRecipePersonNumber{2}
\generalRecipeInfos{}

\ingredienti%
    {Zucchine}{4}%
    {olio extravergine d'oliva}{\qb{}}%
    {erba a scielta (maggiorana, prezzemolo)}{\qb{}}%
    {sale}{\qb{}}%

\begin{enumerate}
    \item Tagliare gli estremi di ogtni zucchina;
    \item Tagliare in parti di 4, massimo 5cm la zucchina in modo da ottenere dei cilindri;
    \item Tagliare ogni cilindro di zucchina in sottili sezioni, di 2-3 mm: più sottili i clindri di zucchina vengono tagliati, più veloce e più semplice sarà la ricetta;
    \item prendere una padella antiaderente e scaldarla (\textbf{non} mettere l'olio!);
    \item una volta che la padella sarà calda inserire tutte le sezioni di succhina in modo che non si sovrappongano;
    \item Lasciarle cuocere finché non sono pronte, quindi girarle per farle cuocere dall'altro lato. E' possibile capire se una sezione di zucchina è pronta o meno se il suo aspetto è simile al seguente:
        \insertImage{src/imgs/zucchine-grigliate/01}{0.4}
    \item Ognio volta che una zucchina è pronta toglierla dalla padella ed inserirla in un piatto;
    \item Una volta finito, spolverarle con del sale, un filo d'olio ed un'erba a vostra scelta (la mamma consiglia prezzemolo, timo, maggiorana, originano);
\end{enumerate}


    \section{Paella con il pesce}

\setRecipePersonNumber{5}
\generalRecipeInfos{}

\ingredienti%
    {Peperone giallo (piccolo)}{\# 1}%
    {Peperone rosso}{\# 1}%
    {Peperone verde}{\# 1}%
    {piccoli molluschi e piccoli crostacei(\eg{} anelli, gamberetti, moscardini)}{2 Kg}%
    {grandi crostacei (\eg{} gamberoni)}{\# 10}%
    {cozze, vongole}{500g}%
    {sale}{\qb{}}%
    {zafferano}{1 bustina}%
    {riso per risotto al dente}{10 pugni}%
    {prezzemolo}{50g}%
    {aglio}{4 spicchi}%
    {olio extra vergine d'oliva}{\qb{}}%
    {pomodorini}{\# 15}%
    {polvere di peperoncino}{\qb{}}%
    {vino di bianco}{120g}%
    {acqua}{3 caraffe}%
    {piselli}{0.5 busta}%
    {cipolla (opzionale)}{\# 0.5}%

In realtà per molti ingredienti esistono delle formule.
Sia:

\begin{itemize}
    \item $|C|$ il numero di commensali;
    \item $g_m$ il peso di piccoli molluschi e di piccoli crostacei da usare;
    \item $g_C$ il peso di cozze e vongole;
    \item $p_{riso}$ il numero di pugni di riso da mettere nella paella;
    \item $T_{acqua}$ il numero di tazze d'acqua da mettere nella paella;
\end{itemize}

Allora:

$$g_m + g_C = 500 \times |C|$$
$$p_{riso} = 2 \times |C|$$
$$T_{acqua} = 2 \times |C| + 1$$


La ricetta è stata ispirata da un prospetto informativo contenente la ricetta della paella\cite{bialetti-1999a}.

\subsection{Preparazione Ingrediente}

La paella è una piatto la cui preparazione di ogni ingrediente è consigliata farla prima di preparare il piatto reale.

\begin{itemize}
    \item tagliare ogni peperone in striscie (togliendone i semi);
    \item mettere i peperoni verdi in un piatto ed i gialli e rossi in un altro;
        \insertImage{src/imgs/paella/preparazione05}{0.4}
    \item sciacquare i piccoli molluschi ed i piccoli crostacei e metterli in un altro piatto;
        \insertImage{src/imgs/paella/preparazione01}{0.4}
    \item sciacquare i grandi crostacei (\eg{} gamberoni) e metterli in un piatto;
        \insertImage{src/imgs/paella/preparazione03}{0.4}
    \item Sciacquare le cozze e metterle da parte. Per le vongole: se sono già state spurgate sciacquarle e metterle insieme alle cozze. Altrimenti immergerle in un lavabo di acqua (dimensioni circa 50 cm $\times$ 50cm  $\times $ 20cm) con un pugno di sale, quindi aspettare 2 ore affinché le vongole rilascino la sabbia;
        \insertImage{src/imgs/paella/preparazione04}{0.4}
    \item tagliare grossolanamente il prezzemolo e schiacciare l'aglio, quindi metterli insieme su un piatto;
    \item prendere una tazza d'acqua ed inserirci la busta di zafferano;
    \item OPZIONALE: tagliare la cipolla a rondelle e metterla da parte;
\end{itemize}

\subsection{Cucina}

\begin{enumerate}
    \item prendere una paellera. Assicurarsi che la paellera sia più a bolla possibile: paellere non a bolla aumentaeranno la difficoltà di cottura, soprattutto nelle prime fasi;
        \insertImage{src/imgs/paella/01}{0.4}
    \item versare l'olio nella paellera, aspettare che l'olio si riscaldi. Se si si vuole è possibile aggiungere la cipolla per fare un soffritto;
        \insertImage{src/imgs/paella/02}{0.4}
    \item aggiungere i piccoli molluschi ed i piccoli crostacei;
    \item Lasciar cuocere per 5 minuti;
        \insertImage{src/imgs/paella/03}{0.4}
    \item incorporare l'aglio, il prezzemolo, i pomodorini ed i peperoni; 
        \insertImage{src/imgs/paella/05}{0.4}
    \item Lasciar cuocere per 10 minuti;
    \item aggiungere i piselli;
        \insertImage{src/imgs/paella/06}{0.4}
    \item Aggiungere il riso;
        \insertImage{src/imgs/paella/09}{0.4}
    \item Aggiungere il vino per farlo sfumare;
        \insertImage{src/imgs/paella/10}{0.4}
    \item Quando il vino è sfumato, aggiungere l'acqua e lo zafferano;
    \item Aggiungere il sale per il riso;
    \item Aspettare 10 minuti;
    \item Aggiungere i gamberoni;
        \insertImage{src/imgs/paella/07}{0.4}
    \item Dopo 5 minuti, aggiungere le cozze;
        \insertImage{src/imgs/paella/12}{0.4}
    \item Aspettare 10 minuti, quindi servire;
        \insertImage{src/imgs/paella/13}{0.4}
\end{enumerate}



	\clearpage
	\bibliographystyle{plain}
	\bibliography{src/bibs/biblio}
\end{document}
