\section{Brasato con polenta}

\setRecipePersonNumber{4}
\generalRecipeInfos{}

\ingredienti%
    {chiodi di garofano}{4 o 6 chiodi}{alloro}{4 o 6 foglie}{sale}{quanto basta}%
    {funghi}{600 grammi di funghi tra chiodini e porcini}%
    {vino rosso}{per sfumare}%
    {carota}{2}%
    {sedano}{2 gambo}%
    {cipolla}{1}%
    {polpa di pomodoro}{4 bicchieri (o 20 cucchiai)}%
    {olio oliva extra vergine}{quanto basta}

\begin{enumerate}

    \item Tritare le carote, le cipolle ed il sedano e metterle da parte;
    \item Mettere  mezzo bicchiere d'olio d'oliva in una capiente pentola di terracotta;
    \item prendere il brasato ed infarinarlo; quindi metterlo in pentola;
    \item inserire mezzo trito di carote, cipolle e sedano;
    \item accendere la fiamma a fuoco alto e lasciar rosolare la carna;
    \item appena rosolato, sfumare il tutto con del vino rosso;
    \item sumato il vino, inserire 2 bicchieri di polpa di pomodoro, 2 chiodi di garofano, 2 foglie di alloro tagliate a metà (in modo che rilascino più sapore), e tanta acqua fino a \textbf{quasi} coprire l'intero brasato (il brasato dovrebbe rimanere fuodi dall'acqua appena appena). Inserire quindi 2 pizzichi di sale ognuno lungo un intero giro di pentola. Infine coprire la pentola con il coperchio (sempre in terracotta); 
    \item 
        Aspettare che si cuocia. Questo processo può durare fino a 2 ore e mezza;
        \insertImage{src/imgs/brasato/06}{0.4}
    \item quando la carne è a metà cottura, togliere il brasato, quindi tagliarlo a fettine di spessore 0.8 - 1 cm. Lo si fa per farla cuocere meglio e per rilasciare più sapore. Per capire quando la carne è a metà cottura bisogna che la carne sia bianca, sugosa e fibrosa all'interno. Dio solito serve un'ora per essere a metà cottura;
    \item 
        nel frattempo prendere un'altra padella, e far rosolare l'altra metà di trito con dell'olio;
        \insertImage{src/imgs/brasato/02}{0.4}
    \item 
        inserire i funghi sgocciolati (non devono essere secchi ma nemmeno carichi di acqua) nella padella, insieme a sale, 2 bicchieri di pomodoro, 2 foglie di alloro tagliate a metà, 2 chiodi di garofano ed una sfilata di olio. Coprire con il coperchio e far cuocere per 15- 20 minuti;
        \insertImage{src/imgs/brasato/07}{0.4}
    \item Il brasato deve avere del sughetto che è ottenuto dall'acqua. Quando l'acqua sta per arrivare al livello corretto di sughetto (che lo decidi te quanto "sugoso" lo vuoi) inserire i funghi;
    \item aspettare 10 minuti;
\end{enumerate}
