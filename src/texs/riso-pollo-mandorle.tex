\section{Riso Pollo e Mandorle}

\setRecipePersonNumber{2}
\generalRecipeInfos{}

\ingredienti%
    {Riso Basmati}{}%
    {Petto di Pollo}{400g}%
    {sale}{\qb{}}%
    {pepe}{\qb{}}%
    {olio extravergine d'oliva}{\qb{}}%
    {miele}{\qb{}}%
    {salsa di soia}{\qb{}}%
    {succo di limone}{\qb{}}%
    {zucchina grande}{1}%
    {carote}{2}%
    {funghi}{\qb{}}%
    {peperoni}{\qb{}}%
    {Mandorle}{\qb{}}%

\note{la ricetta coinvolge 3 fasi in parallelo: cottura del riso, cottura del pollo e cuttura delle verdure.}

\subsection{Cottura dell'acqua}

\begin{enumerate}
    \item far bollire dell'acqua e metterci il sale;
    \item Butta il riso nell'acqua;
\end{enumerate}

\subsection{Pollo}

\begin{enumerate}
    \item Prendere una padella ed inserire il pollo a pezzi di 1-2 cm insieme a sale, olio e pepe;
    \item Una volta che il pollo è cotto aggiungere il 2-3 cucchiaini di miele, 2-3 cucchiaini di soia ed 1 cucchiaini di limone;
\end{enumerate}

\subsection{Verdure}

\begin{enumerate}
    \item Prendere una padella, oliarla ed inserire tutte le verdure considerata (zucchine, peperoni, mandorle); Non inserire il sale poiché esso rovinerebbe le zucchine;
    \item Una volta che le verdure sono cotte aggiungere in padelle sale, pepe, miele, salsa di soia e limone;
    \item mischia il tutto nella padella;
    \item aggiungi mezzo bicchiere d'acqua per far finire la cottura delle verdure;
    \item Aspetta che l'acqua si asciughi;
\end{enumerate}

\subsection{Fase finale}

\begin{enumerate}
    \item Una volta che l'acqua delle verdure si è asciugata unisci le verdure con il pollo e falli andare insieme per alcuni minuti;
    \item Unisci il composto con il riso;
    \item Servire il tutto;
\end{enumerate}


