\section{Paella con il pesce}

\setRecipePersonNumber{5}
\generalRecipeInfos{}

\ingredienti%
    {Peperone giallo (piccolo)}{\# 1}%
    {Peperone rosso}{\# 1}%
    {Peperone verde}{\# 1}%
    {piccoli molluschi e piccoli crostacei(\eg{} anelli, gamberetti, moscardini)}{2 Kg}%
    {grandi crostacei (\eg{} gamberoni)}{\# 10}%
    {cozze, vongole}{500g}%
    {sale}{\qb{}}%
    {zafferano}{1 bustina}%
    {riso per risotto al dente}{10 pugni}%
    {prezzemolo}{50g}%
    {aglio}{4 spicchi}%
    {olio extra vergine d'oliva}{\qb{}}%
    {pomodorini}{\# 15}%
    {polvere di peperoncino}{\qb{}}%
    {vino di bianco}{120g}%
    {acqua}{3 caraffe}%
    {piselli}{0.5 busta}%
    {cipolla (opzionale)}{\# 0.5}%

In realtà per molti ingredienti esistono delle formule.
Sia:

\begin{itemize}
    \item $|C|$ il numero di commensali;
    \item $g_m$ il peso di piccoli molluschi e di piccoli crostacei da usare;
    \item $g_C$ il peso di cozze e vongole;
    \item $p_{riso}$ il numero di pugni di riso da mettere nella paella;
    \item $T_{acqua}$ il numero di tazze d'acqua da mettere nella paella;
\end{itemize}

Allora:

$$g_m + g_C = 500 \times |C|$$
$$p_{riso} = 2 \times |C|$$
$$T_{acqua} = 2 \times |C| + 1$$


\subsection{Preparazione Ingrediente}

La paella è una piatto la cui preparazione di ogni ingrediente è consigliata farla prima di preparare il piatto reale.

\begin{itemize}
    \item tagliare ogni peperone in striscie (togliendone i semi);
    \item mettere i peperoni verdi in un piatto ed i gialli e rossi in un altro;
        \insertImage{src/imgs/paella/preparazione05}{0.4}
    \item sciacquare i piccoli molluschi ed i piccoli crostacei e metterli in un altro piatto;
        \insertImage{src/imgs/paella/preparazione01}{0.4}
    \item sciacquare i grandi crostacei (\eg{} gamberoni) e metterli in un piatto;
        \insertImage{src/imgs/paella/preparazione03}{0.4}
    \item Sciacquare le cozze e metterle da parte. Per le vongole: se sono già state spurgate sciacquarle e metterle insieme alle cozze. Altrimenti immergerle in un lavabo di acqua (dimensioni circa 50 cm $\times$ 50cm  $\times $ 20cm) con un pugno di sale, quindi aspettare 2 ore affinché le vongole rilascino la sabbia;
        \insertImage{src/imgs/paella/preparazione04}{0.4}
    \item tagliare grossolanamente il prezzemolo e schiacciare l'aglio, quindi metterli insieme su un piatto;
    \item prendere una tazza d'acqua ed inserirci la busta di zafferano;
    \item OPZIONALE: tagliare la cipolla a rondelle e metterla da parte;
\end{itemize}

\subsection{Cucina}

\begin{enumerate}
    \item prendere una paellera. Assicurarsi che la paellera sia più a bolla possibile: paellere non a bolla aumentaeranno la difficoltà di cottura, soprattutto nelle prime fasi;
        \insertImage{src/imgs/paella/01}{0.4}
    \item versare l'olio nella paellera, aspettare che l'olio si riscaldi. Se si si vuole è possibile aggiungere la cipolla per fare un soffritto;
        \insertImage{src/imgs/paella/02}{0.4}
    \item aggiungere i piccoli molluschi ed i piccoli crostacei;
    \item Lasciar cuocere per 5 minuti;
        \insertImage{src/imgs/paella/03}{0.4}
    \item incorporare l'aglio, il prezzemolo, i pomodorini ed i peperoni; 
        \insertImage{src/imgs/paella/05}{0.4}
    \item Lasciar cuocere per 10 minuti;
    \item aggiungere i piselli;
        \insertImage{src/imgs/paella/06}{0.4}
    \item Aggiungere il riso;
        \insertImage{src/imgs/paella/09}{0.4}
    \item Aggiungere il vino per farlo sfumare;
        \insertImage{src/imgs/paella/10}{0.4}
    \item Quando il vino è sfumato, aggiungere l'acqua e lo zafferano;
    \item Aggiungere il sale per il riso;
    \item Aspettare 10 minuti;
    \item Aggiungere i gamberoni;
        \insertImage{src/imgs/paella/07}{0.4}
    \item Dopo 5 minuti, aggiungere le cozze;
        \insertImage{src/imgs/paella/12}{0.4}
    \item Aspettare 10 minuti, quindi servire;
        \insertImage{src/imgs/paella/13}{0.4}
\end{enumerate}

