\section{Limoncello}

\setRecipePersonNumber{2}
\generalRecipeInfos{}

\ingredienti%
    {Alcol 90\%}{250g}%
    {Scorza edibile di limoni}{100g}%
    {zucchero}{125g}%
    {acqua}{250g}%

\warning{La ricetta dura nelle fasi iniziali da un giorno a 14 giorni.}

In realtà le dosi hanno delle proporzioni fisse\cite{bressanini-2018a}. Se $g_{scorza}$ sono i grammi di scorza edibile ricavati dai limoni (in media un limone genera 40g di scorza edibile), allora:

$$g_{alcol} = 2.5 \times g_{scorza}$$
$$g_{zucchero} = 0.5 \times g_{alcol}$$
$$g_{acqua} = g_{alcol}$$

\begin{enumerate}
	\item sciacquare i limoni (per evitare che impurità entrino nel limoncello);
	\item tagliare le scorze usando un pela patate facendo attenzione a non incorporare l'albedo (parte bianca); Per togliere ancora più albedo puoi usare un coltello ed abrasare via l'albedo sulla scorza;
	\item tagliare le striscie di scorza in quadratine per migliorare le reazioni chimiche successive;
	\item inserire la scorza in una bottiglia di vetro, quindi aggiungere l'alcol; conservare quindi il composto per un 1 giorno in un posto lontano dalla luce; In realtà
		questa attesa può durare da 1 ad 3 giorni: per un limoncello pù intenso e profumato scegliere 1 giorno, per un limoncello più mordibo 3 giorni (i limoncelli commerciali solitamente usano 3 giorni);
		In generale meglio 1 giorno;
	\item dopo l'attesa riprendere la bottiglia di limoncello, aggiungere zucchero e mischiare;
	\item aggiungere quindi subito dopo l'acqua;

		\insertImage{src/imgs/limoncello/01}{0.4}
	\item volendo è possibile berlo subito; tuttavia per avere un limoncello con più sapori aspettare da 1 a 3 settimane;
\end{enumerate}

