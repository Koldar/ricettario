\section{Zucchine grigliate}

\setRecipePersonNumber{2}
\generalRecipeInfos{}

\ingredienti%
    {Zucchine}{4}%
    {olio extravergine d'oliva}{\qb{}}%
    {erba a scielta (maggiorana, prezzemolo)}{\qb{}}%
    {sale}{\qb{}}%

\begin{enumerate}
    \item Tagliare gli estremi di ogtni zucchina;
    \item Tagliare in parti di 4, massimo 5cm la zucchina in modo da ottenere dei cilindri;
    \item Tagliare ogni cilindro di zucchina in sottili sezioni, di 2-3 mm: più sottili i clindri di zucchina vengono tagliati, più veloce e più semplice sarà la ricetta;
    \item prendere una padella antiaderente e scaldarla (\textbf{non} mettere l'olio!);
    \item una volta che la padella sarà calda inserire tutte le sezioni di succhina in modo che non si sovrappongano;
    \item Lasciarle cuocere finché non sono pronte, quindi girarle per farle cuocere dall'altro lato. E' possibile capire se una sezione di zucchina è pronta o meno se il suo aspetto è simile al seguente:
        \insertImage{src/imgs/zucchine-grigliate/01}{0.4}
    \item Ognio volta che una zucchina è pronta toglierla dalla padella ed inserirla in un piatto;
    \item Una volta finito, spolverarle con del sale, un filo d'olio ed un'erba a vostra scelta (la mamma consiglia prezzemolo, timo, maggiorana, originano);
\end{enumerate}

