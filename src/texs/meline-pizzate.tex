\section{Meline pizzate}

\setRecipePersonNumber{1}
\generalRecipeInfos{}

\ingredienti%
    {funghi Pleurotus (o \quotes{melina})}{4}%
    {sale}{\qb{}}%
    {pepe}{\qb{}}%
    {origano}{\qb{}}%
    {mozzarella}{1}%
    {passata o salsa di pomodoro}{6 cucchiai}%

\note{Se non hai i funghi Pleurotus, puoi facilmente sostituirli con mezza melanzana: è un'ottima alternativa.}

\begin{enumerate}
    \item Prendo i funghi Pleurotus e li lavo. Se invece sto usando la melanzana la lavo e la taglio ortogonale all'asse (ossia in rondelle) in fette ognuna circa 0.8cm o 1.0cm;

    \item accendere il forno e aspettare che arriva a 200/220 °C;
    \item inserire i funghi o le melanzane nel forno mettendoli in una padella con della carta forno;

    \item aspettare circa 3/5 minuti: in realtà è necessario aspettare che si \quotes{ringrinziscano}. \`E posibile capirlo quando sono pronti quando le verdure incominciano ad assumere in alcune parti del colore marroncino;

    \item tirare fuori i funghi (o le melanzane), quindi mettere un cucchiaio e mezzo di salsa(passata) di pomodofo su ciascun pezzo;

    \item aggiungere sale e pepe quindi spalmare con un cucchiaio il pomodoro su ciascun ortaggio (come se fosse una pizza);

    \item aggiungere la mozzarella tagliata a cubetti (dimensioni circa $0.5cm^3$) e spargerla su ciscuna fetta;

    \item aggiungere l'origano;

    \item reinserire i funghi (o le melanzane) in forno ed aspettare finché la mozzarella non si è sciolta: ci può impiegare circa 10/12 minuti;
    \insertImage{src/imgs/meline-pizzate/01}{0.4}

    \item Togliere e servire in tavola;
    \insertImage{src/imgs/meline-pizzate/02}{0.4}
    
\end{enumerate}
